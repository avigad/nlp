\documentclass{article}

\usepackage{amsmath, amsthm}

\newtheorem{theorem}{Theorem}[section]
\newtheorem{lemma}[theorem]{Lemma}
\newtheorem{proposition}[theorem]{Proposition}
\newtheorem{corollary}[theorem]{Corollary}
\newtheorem{definition}[theorem]{Definition}

\newcommand{\block}{{\sf [block]}}
\newcommand{\blockend}{{\sf [block-end]}}

\begin{document}
 
\begin{proposition} 
Let $A$, $B$, and $C$ be sets. Then $A \cup (B \cap C) = (A \cup B) \cap (A 
\cup C)$.
\end{proposition}

\begin{proof}
Suppose $x$ is in $A \cup (B \cap C)$. Then either $x$ is in $A$, or it is in 
$B \cap C$. In the first case, since $x$ is in $A$, it is in both $A \cup B$ and 
$A \cup C$, and hence in $(A \cup B) \cap (A \cup C)$. In the second case, $x$ 
is in both $B$ and $C$. This means that once again it is in both $A \cup B$ and 
$A \cup C$, and hence in $(A \cup B) \cap (A \cup C)$.

For the other direction, suppose $x$ is in $(A \cup B) \cap (A \cup C)$. Then 
it is in both $A \cup B$ and $A \cup C$. If $x$ is in $A$, then it is in $A \cup 
(B \cap C)$, as required. If $x$ is not in $A$, then the fact that it is in $A 
\cup B$ implies that it is in $B$, and the fact that it is in $A \cup C$ implies 
that it is in $C$. Hence it is in $B \cap C$, and hence $A \cup (B \cap C)$, as 
required.
\end{proof}

Here is a version that shows the block structure.

\begin{proof}\ 
\begin{quote}
Suppose $x$ is in $A \cup (B \cap C)$. Then either $x$ is in $A$, or it is in $B 
\cap C$. 
\begin{quote}
In the first case, since $x$ is in $A$, it is in both $A \cup B$ and 
$A \cup C$, and hence in $(A \cup B) \cap (A \cup C)$. 
\end{quote}
\begin{quote}
In the second case, $x$ 
is in both $B$ and $C$. This means that once again it is in both $A \cup B$ and 
$A \cup C$, and hence in $(A \cup B) \cap (A \cup C)$.
\end{quote}
\end{quote}

\begin{quote}
For the other direction, suppose $x$ is in $(A \cup B) \cap (A \cup C)$. Then it 
is in both $A \cup B$ and $A \cup C$. 
\begin{quote}
If $x$ is in $A$, then it is in $A \cup (B \cap C)$, as required. 
\end{quote}
\begin{quote}
If $x$ is not in $A$, then the fact that it is in $A \cup B$ implies that it is 
in $B$, and the fact that it is in $A \cup C$ implies that it is in $C$. Hence 
it is in $B \cap C$, and hence $A \cup (B \cap C)$, as required. 
\end{quote}
\end{quote}
\end{proof}

\begin{proposition} 
Let $A$, $B$, and $C$ be sets. Then $A \cap (B \cup C) = (A \cap B) \cup (A 
\cap C)$.
\end{proposition}

\begin{proof}
Suppose $x$ is in $A \cap (B \cup C)$. Then $x$ is in $A$, and $x$ is in $B 
\cup C$. This means that $x$ is in either $B$ or $C$. If $x$ is in $B$, it is 
in 
$A \cap B$, and hence $(A \cap B) \cup (A \cap C)$. If $x$ is in $C$, it is in 
$A \cap C$, and hence $(A \cap B) \cup (A \cap C)$. Either way, we have the 
desired conclusion.

Conversely, suppose $x$ is in $(A \cap B) \cup (A \cap C)$. If $x$ is in $A 
\cap B$, it is in both $A$ and $B$. Hence $x$ is in $B \cup C$, and $A \cap (B 
\cup C)$. If $x$ is in $A \cap C$, it is in both $A$ and $C$. Hence $x$ is in 
$B 
\cup C$, and again in $A \cap (B \cup C)$, as required.
\end{proof}

\begin{proposition}
Let $A$ and $B$ be sets. Then $\overline{A \setminus B} = \overline{A} \cup B$.
\end{proposition}

\begin{proof}
Suppose $x$ is in $\overline{A \setminus B}$. Then $x$ is not in $A \setminus 
B$. We want to show that $x$ is in $\overline{A} \cup B$. If $x$ is in 
$\overline{A}$, we are done. Otherwise, $x$ is in $A$. Then, if $x$ is not in 
$B$, then $x$ is in $A \setminus B$, a contradiction. So $x$ is in $B$, and so 
it is in $\overline{A} \cup B$.

Conversely, suppose $x$ is in $\overline{A} \cup B$. Then either $x$ is not in 
$A$, or $x$ is in $B$. If $x$ is not in $A$, then it is not in $A \setminus B$, 
as required. If $x$ is in $B$, then it is also not in $A \setminus B$, as 
required.
\end{proof}

\begin{proposition}
$(A \setminus B) \setminus C = A \setminus (B \cup C)$.
 \end{proposition}

\begin{proof}
Suppose $x$ is in $(A \setminus B) \setminus C$. Then $x$ is
in $A \setminus B$ but not $C$, and hence it is in $A$ but not $B$ or
$C$. This means that $x$ is in $A$ but not $B \cup C$, and so in $A
\setminus (B \cup C)$.

Conversely, suppose $x$ is in $A \setminus (B \cup C)$. Then $x$ is in
$A$, but not in $B \cup C$. In particular, $x$ is in neither $B$ nor
$C$, because otherwise it would be in $B \cup C$. So $x$ is in $A
\setminus B$, and hence $(A \setminus B) \setminus C$.
\end{proof}

\begin{definition}
Two sets $A$ and $B$ are \emph{disjoint} if $A \cap B = \emptyset$.
\end{definition}

\begin{proposition}
Show that If $A$ and $B$ are disjoint, $C \subseteq A$, and $D \subseteq B$, 
then $C$ and $D$ are disjoint. 
\end{proposition}

\begin{proof}
Suppose $A$ and $B$ are disjoint, $C \subseteq A$, and $D \subseteq B$. We need 
to show that there is no element $x$ that is in both $C$ and $D$. Suppose 
otherwise, that is, $x$ is in $C$ and $x$ is in $D$. Since $C \subseteq A$, we 
have that $x$ is in $A$, and since $D \subseteq B$, we have that $x$ is in $B$. 
So $x$ is in $A$ and $B$, contradicting the fact that $A$ and $B$ are disjoint.
\end{proof}



\end{document}
